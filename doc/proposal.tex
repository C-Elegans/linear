\documentclass{article}
\usepackage[margin=1.5in]{geometry}
\usepackage[nottoc,numbib]{tocbibind} 
\usepackage{hyperref}
\title{Using Linear Types to bring Functional Programming to Embedded Systems}
\author{Michael Nolan}
\begin{document}
\maketitle
\tableofcontents
\newpage
\section{Abstract}
Abstract here
\section{Purpose}
To develop a compiler that incorporates linear types in its compilation strategy to 
reduce the need for a garbage collector.
\section{Summary}

Statically typed functional programming languages offer better safety
guarantees and reduce programmer errors better than almost all other
programming languages. These languages, previously relegated to use by
academics for PhD theses, are now being used by businesses because of their
greater reliability and rapid development. Unfortunately, functional
programming languages generate a large amount of garbage and do not allow the
programmer to allocate memory on their own. This means that they are unsuitable
for use in embedded systems, where realtime guarantees must be made and there
is little memory availible for use. 

However, the addition of linear types\cite{girard} to the language allows the
programmer to tell the compiler that certain variables will be used
\textit{exactly once}. This means that the compiler can free the variable as it
is used and be sure that memory will not be leaked or freed twice. Additionally, 
the compiler can fuse computations\cite{ghc-linear,tweag-blog}, meaning that intermediate
values do not need to be constructed. Finally, linear types provide additional safety
when used properly, guaranteeing that an action happens exactly once, which is useful
in a client-server architecture.

\section{Methodology}
\newpage
\bibliographystyle{ieeetr}
\bibliography{bib}
\end{document}
